\begin{abstract}
    少人数のソフトウェア開発において,ビジネス要求を素早く実現するためにソースコードの追加・修正に追われ,
    ドキュメントの追加・修正を忘れてしまい,ソフトウェア品質を落としてしまうといった状況が見られる.
    これは,ソースコードからドキュメントを自動生成する技術を使用したり,ドキュメントとソースコードが乖離したときに開発者に通知を行うことで対応ができる.
    しかし,ソースコードからドキュメントを自動生成するためには,ライブラリやフレームワークに依存してしまうといった問題点があり,
    ドキュメントとソースコードが乖離したときに通知を行うためには手間がかかる問題点がある.
    
    以上より,ライブラリやフレームワーク,エディタに依存しない,ドキュメントとソースコードの乖離の可能性のある箇所の特定および通知を行うことのできるツールを開発した.
    ツールを利用するためには,CI/CDツールの設定ファイルに数行追加するだけでよいため,非常に軽量である.
    また,ツールの有効性を検証するために,Gitのコミット履歴を用いたシミュレータの開発および評価実験を行った.
    Pythonが使用されたOSSのプロジェクトを対象にシミュレーションを行ったところ,乖離リスクのある箇所に対して通知を行うといった結果が得られた.
    評価実験では,実験協力者2名に対し,前半5個,後半5個の合計10個のタスクをこなしてもらい,前後半のいずれかでツールが作動する実験を行った.
    このとき,ドキュメントとソースコードの乖離が抑制されるのかを評価の対象とした.
    また,実験終了後にアンケートを実施した.
    実験の結果,実験終了時にはドキュメントとソースコードが乖離することはなく,一定の効果が見込めることが判明した.
    また,アンケート結果から,「ドキュメントの更新を忘れてしまったとき,通知は必要である」といった回答を得ており,
    ドキュメントとソースコードの乖離抑制に役立つことができたと考えられる.
\end{abstract}