\chapter{序論}
近年のソフトウェア開発では,ソフトウェア開発における複雑性が増し,少人数の開発チームへと分割されていく中で,開発者は短期間でより多くのを作業をしなくてはらならない状況にある.
そのため,分割されたチーム間で円滑な連携を行うことや,ソフトウェア品質を保つためにドキュメントを整備することが重要視されている.
また,2020年初旬からの新型コロナウィルス感染症拡大により,ソフトウェア開発の現場ではリモートワークを中心としたテキストベースでのやり取りも増えたため,より一層ドキュメントを充実させる必要性が高まった.
このような状況において,開発スピードを求められるソフトウェア開発の現場では,ドキュメントとソースコードの整合性を維持することが困難となってきた.

ソフトウェア開発の現場では,バージョン管理システムを使用したソフトウェアのバージョン管理,CI/CD(継続的インテグレーション/継続的デリバリー)を活用したソフトウェアのビルド・テスト・デプロイなどの自動化,
円滑なやりとりを行うためのチャットツールの導入など,開発スピードの向上に力を入れている.
また,同じ開発チームであったとしても,フロントエンドの開発,バックエンドの開発,モバイルアプリケーションの開発,デザインの調整など開発者の役割は様々であり,
開発チームの内外問わずに円滑なやりとりを行う必要がある.
このため,ソフトウェア開発で使用するドキュメントは常に最新のものであることが求められている.

開発者は,ビジネス要求を素早く実現するために,機能の追加や修正を優先してしまい,ドキュメントの更新を後回しにしてしまう事例は少なくない.
例えば,誤って古いドキュメントを参照して開発を進めた場合,後の段階で手戻りが発生していしまうため,開発スピードを大幅に落としてしまう原因となってしまう.
また,普段と異なる開発環境では開発者は自身の能力を最大限発揮することは難しい.
したがって,ライブラリやフレームワーク,エディタに依存せずに,既存のソフトウェア開発プロジェクトに無理なく組み込むことができるツールを作成する必要があると考えた.

以上を踏まえた上で,ドキュメントとソースコードが乖離している可能性のある箇所を特定および開発者に通知をすることができる機能をもった支援ツールmilkを開発した.
milkはライブラリやフレームワーク,エディタに依存しないツールである.
同時に,milkの有用性を検証するために,Gitのコミット履歴を使用したシミュレータの開発を行い,実際のソフトウェア開発を想定したシミュレーションを行った.
シミュレーションで検証できなかった機能については,実際にプログラミング経験が豊富な実験協力者にmilkを使用してもらい,その有用性を確かめる評価実験を行った.

本論文の構成は以下のとおりである.2章で,本研究の関連研究について示し,本研究との相違点を述べる.
3章では本研究で提案するツールについて,4章ではその設計と実装について述べる.
5章では,提案ツールの有用性検証するためのシミュレーションおよび評価実験について述べる.
6章では,評価実験の結果と考察について述べ,7章で本研究の結論を述べる.