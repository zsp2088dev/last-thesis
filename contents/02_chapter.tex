\chapter{関連研究}
本章では,本研究に関連する研究および技術と,本研究との相違点について述べる.

\section{ドキュメントとソースコードの一元管理}
赤石らの提案するドキュメントとソースコードの一元管理ツールは,開発者がXML形式のドキュメントを作成した後,それに従ってコーディングを行い,ソースコードを作成する
作成したドキュメントとソースコードは,ビュワーを通じて閲覧を行うことができる.ビュワーではドキュメントとソースコードが混在した形で近い場所に表示されるため,ソースコードの変更に気づきやすい.
しかし,ドキュメントとソースコードの整合性を維持するためには,赤石らの提案するエディタおよびビュワーを使用しなくてはいけないため,開発者の学習コストや開発スピードの低下などが懸念される.

\section{ドキュメント自動生成}

\section{本研究との相違点}

赤石らの研究で提案されたドキュメントとソースコードの一元管理を行うツールでは,目的は同じであるものの,プログラミング言語やエディタが限定されるのに対し,
本研究では,ライブラリやフレームワーク,エディタに依存せずに,既存のソフトウェア開発プロジェクトに無理なく組み込むことができるといった点で異なる.
ドキュメントを自動生成するライブラリやフレームワークでは,技術選定の段階で制限されてしまうのに対し,本研究では,現在進行中のプロジェクトに無理なく組み込むことができるといった点で異なる.

本研究ではドキュメントとソースコードが乖離している可能性のある箇所を特定および開発者に通知をすることができる機能をもったツールの開発を行う.
