\chapter{関連研究}
本章では,本研究に関連する研究および技術と,本研究との相違点について述べる.

\section{ドキュメントとソースコードの一元管理}
赤石らの提案するドキュメントとソースコードの一元管理ツールは,開発者がXML形式のドキュメントを作成した後,それに従ってソースコードを作成することで,ドキュメントとソースコードの整合性を維持すことを期待する.
作成したドキュメントとソースコードは,ビュワーを通じて閲覧を行うことができる.ビュワーではドキュメントとソースコードが混在した形で近い場所に表示されるため,開発者はドキュメントとソースコードの乖離に気づきやすい.
しかし,ドキュメントとソースコードの整合性を維持するためには,赤石らの提案するエディタおよびビュワーを使用しなくてはいけないため,開発者の学習コストや開発スピードの低下などが懸念される.
また,ドキュメントとソースコードの対応付けを行う方法や,提案ツールの評価,ビュワーの見た目などが未実施である.

\section{ドキュメントとソースコードを対応付ける研究}
後藤らのドキュメントとソースコードの対応付けを行う研究\cite{taiouduke}では,ドキュメントをXML形式で記述したものを扱い,
XMLのタグとソースコード中の属性値を対応表を用いて対応付けを行う.これによって,ドキュメントとソースコードの整合性を検査できるようになった.
しかし,ドキュメントまたはソースコードのどちらかの編集に対して,インタラクティブな機能を提供することができず,
例えば,先にドキュメントを作成してからソースコードを記述する方法では不整合となってしまう問題がある.

\section{ドキュメントからソースコードを生成する研究}
小池のソースコードからソースコードを生成するフレームワークSLAF\cite{framework}では,ドキュメントとソースコードの乖離の発生は防げないといった考えから,
ドキュメントからソースコードを生成することができる.
具体的には,プログラミング言語の知識を必要としないドキュメントを記述することで,それに従って動作するJavaのソースコードが生成される.
これによって,常にドキュメントとソースコードの整合性が維持されることになった.
しかし,生成されたソースコードでは性能低下が見られることや,現実的に運用することが難しいといったことから課題も多い.

海老澤らのWebComponents開発におけるAPIドキュメントとソースコードを自動生成する研究\cite{webcomponents}では,
1つのファイルに,HTML, CSS, JavaScriptとAPIドキュメントを記述することで,ドキュメントとソースコードを同時に管理することのできるシステムを開発した.
これは,既存のシステム開発に組み込むことは難しく,新たなプロジェクトで有効に作用すると考えられる.

\section{ドキュメントを自動生成する技術}
ソースコード中に適切なアノテーションやコメント文を挿入することで,ドキュメントを自動生成することのできるライブラリやフレームワークが存在する.
それぞれのライブラリやフレームワークでドキュメントを自動生成するための書き方は異なるが,いずれも適切な箇所で適切なフォーマットでコーディングする必要がある.
以下では,ドキュメントを自動生成することができるライブラリおよびフレームワークを紹介する.

\subsection{Javadoc}
Javadocは,JDKに標準搭載された機能で,適切なフォーマットに従って作成したコメント文やソースコードからHTML形式のドキュメントを自動生成するツールである.
プログラミング言語にJavaを使用したソフトウェア開発では,ソースコードのAPI仕様書としてにJavadocが使用されることも多い.
Javaのソースコードからドキュメントを生成するため,例えば,プロジェクト全体の方針,アーキテクチャなどの概念図,各機能の設計といった内容のドキュメントを記述する場合は不向きである.

\subsection{Spring Boot}
Spring Bootは,Webアプリケーション開発を行うことのできるJavaのWebフレームワークである.
RESTful APIを実装するときに,Swaggerと呼ばれるドキュメントを自動生成する機能が存在する.
Javaで記述するソースコード中に適切な箇所で適切なアノテーションを埋め込むことで,ソースコードから自動でドキュメントを生成することができる.

\subsection{FastAPI}
FastAPIは,APIサーバーを構築することのできるPythonのWebフレームワークである.
FastAPIにはSwaggerとReDocと呼ばれるドキュメントを自動生成する機能が存在する.
Spring Bootと同様に,ソースコード中に適切な箇所で適切なアノテーションを埋め込むことで,ソースコードから自動でドキュメントを生成することができる.
また,追加のプラグインやライブラリを必要とせずに,元からある機能として組み込まれているため,簡単に利用することができる.
しかし,SwaggerやReDoc以外のドキュメントを生成することができないため,その他ドキュメントに関しては,開発者が意識して,ドキュメントとソースコードの整合性を維持しなくてはならない.

\section{本研究との相違点}
赤石らの研究で提案されたドキュメントとソースコードの一元管理を行うツールでは,目的は同じであるものの,プログラミング言語やエディタが限定されるのに対し,
本研究では,ライブラリやフレームワーク,エディタに依存せずに,既存のソフトウェア開発プロジェクトに無理なく組み込むことができるといった点で異なる.
ドキュメントを自動生成するライブラリやフレームワークでは,技術選定の段階で制限されてしまうのに対し,本研究では,現在進行中のプロジェクトに無理なく組み込むことができるといった点で異なる.

また,ドキュメントを自動生成することができるライブラリやフレームワークでは,ソースコードに直接関係のないドキュメント,例えば,開発プロジェクトのアーキテクチャの概念図やプロジェクトの方向性,
補助的な説明などを記載するドキュメントは,通常ソースコード中には記載せずに,別途ドキュメント専用のファイルなどに記載するため,自動生成がすべてをカバーできないことに留意する.
これらのドキュメントも,プロジェクトが進み古くなってしまうと,ドキュメントとソースコードが乖離してしまう可能性がある.

本研究ではドキュメントとソースコードが乖離している可能性のある箇所を特定および開発者に通知をすることができる機能をもったツールの開発を行う.
