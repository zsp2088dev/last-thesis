\chapter{関連研究}
本章では,本研究に関連する研究および技術と,本研究との相違点について述べる.

\section{ドキュメントとソースコードの一元管理}
赤石らの提案するドキュメントとソースコードの一元管理ツールは,開発者がXML形式のドキュメントを作成した後,それに従ってソースコードを作成することで,ドキュメントとソースコードの整合性を維持すことを期待する.
作成したドキュメントとソースコードは,ビュワーを通じて閲覧を行うことができる.ビュワーではドキュメントとソースコードが混在した形で近い場所に表示されるため,開発者はドキュメントとソースコードの乖離に気づきやすい.
しかし,ドキュメントとソースコードの整合性を維持するためには,赤石らの提案するエディタおよびビュワーを使用しなくてはいけないため,開発者の学習コストや開発スピードの低下などが懸念される.
また,ドキュメントとソースコードの対応付けを行う方法や,提案ツールの評価,ビュワーの見た目などが未実施である.

\section{ドキュメント自動生成}
ソースコード中に適切なアノテーションやコメント文を挿入することで,ドキュメントを自動生成することのできるライブラリやフレームワークが存在する.
それぞれのライブラリやフレームワークでドキュメントを自動生成するための書き方は異なるが,いずれも適切な箇所で適切なフォーマットでコーディングする必要がある.
以下では,ドキュメントを自動生成することができるライブラリおよびフレームワークを紹介する.


\subsection{Javadoc}
Javadocは,JDKに標準搭載された機能で,適切なフォーマットに従って作成したコメント文やソースコードからHTML形式のドキュメントを自動生成するツールである.
プログラミング言語にJavaを使用したソフトウェア開発では,ソースコードのAPI仕様書としてにJavadocが使用されることも多い.

\subsection{Spring Boot}
Spring Bootは,Webアプリケーション開発を行うことのできるJavaのWebフレームワークである.
REST APIを実装するときに,Swaggerと呼ばれるドキュメントを自動生成する機能が存在する.

\subsection{FastAPI}
FastAPIは,APIサーバーを構築することのできるPythonのWebフレームワークである.
FastAPIにはSwaggerとReDocと呼ばれるドキュメントを自動生成する機能が存在する.

\section{本研究との相違点}
赤石らの研究で提案されたドキュメントとソースコードの一元管理を行うツールでは,目的は同じであるものの,プログラミング言語やエディタが限定されるのに対し,
本研究では,ライブラリやフレームワーク,エディタに依存せずに,既存のソフトウェア開発プロジェクトに無理なく組み込むことができるといった点で異なる.
ドキュメントを自動生成するライブラリやフレームワークでは,技術選定の段階で制限されてしまうのに対し,本研究では,現在進行中のプロジェクトに無理なく組み込むことができるといった点で異なる.

ソースコードに直接関係のないドキュメント,例えば,開発プロジェクトのアーキテクチャの概念図やプロジェクトの方向性,補助的な説明などを記載するドキュメントは,通常ソースコード中には記載せずに,
別途ドキュメント専用のファイルなどに記載するため,自動生成がすべてをカバーできないことに留意する.
これらのドキュメントも,プロジェクトが進み古くなってしまうと,ドキュメントとソースコードが乖離してしまう可能性がある.

本研究ではドキュメントとソースコードが乖離している可能性のある箇所を特定および開発者に通知をすることができる機能をもったツールの開発を行う.
