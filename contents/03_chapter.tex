\chapter{提案手法}
本章では,本研究で提案するツールについて述べる.

\subsection{想定利用者}
1章で述べたとおり,ソフトウェア開発のチームでドキュメントとソースコードの乖離を抑制することが目的なため,
ソフトウェア開発を行う少人数(3名〜6名程度)の開発チームを対象としている.
大人数で行うソフトウェア開発では,プロジェクトで使用する技術やツールの自由度が制限されることが多いため,提案ツールは少人数のソフトウェア開発に焦点を当てて開発した.

\section{提案ツール}
本節では,提案ツールの目的,要求仕様について述べる.

\subsection{提案ツールの目的と要求仕様}
提案ツールの役割は,ドキュメントとソースコードの乖離リスクのある箇所または原因を特定し,開発者への通知を行ったときに,通知内容をもとに開発者がドキュメントまたはソースコードを修正することで,乖離を抑制することを期待する.

これを実現するためにツールには以下の機能をもたせる.このとき,それぞれの機能にC1〜C4と名付けた.
いずれの機能も,リモートリポジトリであるGitHub上にプッシュされた際に,CI/CDツール上で動作するmilkが乖離リスクを検知する仕組みとなっている.

\subsection{C1:ソースコード先行の検知の機能}
ドキュメントとソースコードの対応付けを,アノテーションを用いて行い,ドキュメント中に記述されていない機能をソースコードに実装した際に,乖離リスクとして開発者に通知する.
本機能は,RESTful API開発に利用可能であり,プログラミング言語にPython,ドキュメントにSwaggerを選択する必要がある.
開発者は,この通知を受け取ったときに,ドキュメントに適切なものを追加するか,誤ったアノテーションを修正し,ドキュメントとソースコードの乖離を抑制することを期待する.

\subsection{C2:一定時間経過後のリマインダーの機能}
ドキュメントが作成され一定時間経過した後,ソースコードは更新され続けているがドキュメントは更新されていない場合に,乖離リスクとして開発者に通知する.
本機能は,ドキュメントとソースコードが同じリポジトリで管理されることを想定している.
また,Markdown形式ファイルやreStructuredText形式ファイル,単純なテキストファイルなどをドキュメントとして利用する必要がある.
開発者は,この通知を受け取ったときに,ドキュメントが古くなっていないかを確認し,古くなった箇所を修正するか,または現状維持でよいのかを判断し,ドキュメントとソースコードの乖離を抑制することを期待する.

\subsection{C3:リリース時ドキュメント更新有無の検知の機能}
新たなバージョンがリリースされたとき,前回のバージョンリリース時よりドキュメントが更新されていない場合に,乖離リスクとして開発者に通知する.
本機能は,一定時間経過後のリマインダーの機能と同様に,ドキュメントとソースコードが同じリポジトリで管理されていることを想定している.
開発者は,この通知を受け取ったときに,前回のバージョンリリース時との変更点をドキュメントに記述するか,または現状維持でよいのかを判断し,ドキュメントとソースコードの乖離を抑制することを期待する.

\subsection{C4:ソースコード変更量の検知の機能}
新たなファイルを作成,既存のファイルを変更・削除したとき,ある一定量の変更が行われた場合に,乖離リスクとして開発者に通知する.
本機能は,一定時間経過後のリマインダーの機能と同様に,ドキュメントとソースコードが同じリポジトリで管理されていることを想定している.
開発者は,この通知を受け取ったときに,ドキュメントが古くなっていないかを確認し,古くなった箇所を修正するか,または現状維持でよいのかを判断し,ドキュメントとソースコードの乖離を抑制することを期待する.