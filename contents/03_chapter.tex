\chapter{提案手法}
本章では,本研究で提案するツールについて述べる.

\section{想定利用者}
1章で述べたとおり,ソフトウェア開発のチームでドキュメントとソースコードの乖離を抑制することが目的なため,
ソフトウェア開発を行う少人数(3名〜6名程度)の開発チームを対象としている.
大人数で行うソフトウェア開発では,プロジェクトで使用する技術やツールの自由度が制限されることが多いことや,
少人数のソフトウェア開発チームに分割されていくことを踏まえて,提案ツールは少人数のソフトウェア開発に焦点を当てて開発した.

\section{提案ツール}
\label{tool}
本節では,提案ツールの目的,要求仕様について述べる.

\subsection{提案ツールの目的と要求仕様}
\label{required}
提案ツールの役割は,ドキュメントとソースコードの乖離リスクのある箇所を特定し,開発者への通知を行ったときに,通知内容をもとに開発者がドキュメントまたはソースコードを修正することで,乖離を抑制することを期待する.
また,既存のソフトウェア開発に無理なく組み込むことができるようにするために,ライブラリやフレームワーク,エディタへの制限が少なくなるように設計する.
これを実現するためにツールには以下の機能をもたせる.このとき,それぞれの機能にC1〜C4と名付けた.
いずれの機能も,リモートリポジトリであるGitHub上にプッシュされた際に,CI/CDツール上で動作するツールが乖離リスクを検知する仕組みとなっている.
また,C1〜C4の機能について,開発者がパラメータを自由に変更できるようにし,パラメータを変更する際はコミュニケーションツールであるSlackのコマンドを用いて行う.

\subsection{C1:ソースコード先行の検知の機能}
\label{c1}
アノテーションを用いてドキュメントとソースコードの対応付けを行い,ドキュメント中に記述されていない機能をソースコードに実装した際に,乖離リスクとして開発者に通知する.
本機能は,RESTful APIの開発に利用可能であり,プログラミング言語にPython,ドキュメントにSwaggerを選択する必要がある.
開発者は,この通知を受け取ったときに,ドキュメントに適切なものを追加するか,誤ったアノテーションを修正し,ドキュメントとソースコードの乖離を抑制することを期待する.

具体的な仕組みについて,FastAPIを使用したRESTful APIのソースコード\ref{c1code}と,Swaggerのドキュメント\ref{c1doc}を例にまとめる.
これは,GETメソッドで/usersにHTTPリクエストを送信すると,ユーザーの一覧取得を行うことのできるRESTful AIPのソースコードとドキュメントである.
Swagger形式のドキュメントは,JSONまたはYAMLで記述することができる.
ドキュメント中のoperationIdはオプション値であり,エンドポイント毎にユニークな値を設定する必要がある.
また,機能となる関数の上部(FastAPIではデコーレータである@appの上部)に,コメントアウトで@operationIdと記述することで,ドキュメントとソースコードの対応付けを簡単に行えるようになる.

\begin{lstlisting}[caption=RESTful API, label=c1code]
# @operationId get_users
@app.get('/users')
def get_users():
    response = []

    for user in get_users_from_db():
        response.append({
            'user_id': user.user_id,
            'name': user.name,
            'age': user.age
        })

    return response
\end{lstlisting}


\begin{lstlisting}[caption=Swagger,label=c1doc]
"paths": {
     "/users": {
         "get": {
            "summary": "Get a list of users.",
            "operationId": "get_users",
            "responses": {
                "200": {
                    "description": "Successfully get the list of users."
                }
            }
        }
    }
}
\end{lstlisting}

以下に,C1で変更可能なパラメータについてまとめる.
\begin{description}
    \item[C1a:ドキュメントの拡張子] C1機能で利用するSwaggerドキュメントの拡張子(現状では.jsonのみ利用可能)
    \item[C1b:ソースコードの拡張子] C1機能で利用するソースコードの拡張子(現状では.pyのみ利用可能)
\end{description}


\subsection{C2:一定時間経過後のリマインダーの機能}
\label{c2}
ドキュメントが作成され一定時間経過した後,ソースコードは更新され続けているがドキュメントは更新されていない場合に,乖離リスクとして開発者に通知する.
本機能は,ドキュメントとソースコードが同じリポジトリで管理されることを想定している.
また,Markdown形式ファイルやreStructuredText形式ファイル,単純なテキストファイルなどをドキュメントとして利用する必要がある.
開発者は,この通知を受け取ったときに,ドキュメントが古くなっていないかを確認し,古くなった箇所を修正するか,または現状維持でよいのかを判断し,ドキュメントとソースコードの乖離が抑制されることを期待する.

以下に,C2で変更可能なパラメータについてまとめる.
\begin{description}
    \item[C2a:経過日数] ドキュメントが更新されないことを通知するために必要な経過日数
    \item[C2b:ドキュメントの拡張子] C2機能で利用するドキュメントの拡張子(自由に設定可能)
    \item[C2c:ソースコードの拡張子] C2機能で利用するソースコードの拡張子(自由に設定可能)
\end{description}

\subsection{C3:リリース時ドキュメント更新有無の検知の機能}
\label{c3}
新たなバージョンがリリースされたとき,前回のバージョンリリース時よりドキュメントが更新されていない場合に,乖離リスクとして開発者に通知する.
本機能は,一定時間経過後のリマインダーの機能と同様に,ドキュメントとソースコードが同じリポジトリで管理されていることを想定している.
開発者は,この通知を受け取ったときに,前回のバージョンリリース時との変更点をドキュメントに記述するか,または現状維持でよいのかを判断し,ドキュメントとソースコードの乖離を抑制することを期待する.

具体的な仕組みについて,Pythonのパッケージ管理およびバージョン管理を行うソースコード\ref{c3version}のsetup.pyを例にまとめる.
これには,Pythonファイルにプロジェクト名である「management-system」と,そのバージョンは0.0.1が記述されている.
ここに記述されているversionを読み取り,C3の機能として利用する.
その他,例えばJavaScriptやTypeScriptなどのプロジェクトで使用されるpackage.jsonでは,JSONファイルにプロジェクト名やバージョンを記載する.

\begin{lstlisting}[caption=setup.py, label=c3version]
from setuptools import setup

setup(
    name='management-system',
    version='0.0.1'
)    
\end{lstlisting}

以下に,C3で変更可能なパラメータについてまとめる.
\begin{description}
    \item[C3a:バージョン管理を行うファイルのパス] バージョン管理を行うファイルの相対パス(現状ではsetup.pyのみ利用可能)
    \item[C3b:ドキュメントの拡張子] C3機能で利用するドキュメントの拡張子(自由に設定可能)
    \item[C3c:ソースコードの拡張子] C3機能で利用するソースコードの拡張子(自由に設定可能)
\end{description}

\subsection{C4:ソースコード変更量の検知の機能}
\label{c4}
新たなファイルを作成,既存のファイルを変更・削除したとき,ある一定量の変更が行われた場合に,乖離リスクとして開発者に通知する.
本機能は,一定時間経過後のリマインダーの機能と同様に,ドキュメントとソースコードが同じリポジトリで管理されていることを想定している.
開発者は,この通知を受け取ったときに,ドキュメントが古くなっていないかを確認し,古くなった箇所を修正するか,または現状維持でよいのかを判断し,ドキュメントとソースコードの乖離を抑制することを期待する.

以下に,C4で変更可能なパラメータについてまとめる.
\begin{description}
    \item[C4a:ソースコードの変更量] ソースコードの追加・削除・変更によって通知を送るのに必要な変更量
    \item[C4b:ドキュメントの拡張子] C4機能で利用するドキュメントの拡張子(自由に設定可能)
    \item[C4c:ソースコードの拡張子] C4機能で利用するソースコードの拡張子(自由に設定可能)
\end{description}