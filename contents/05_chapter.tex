\chapter{評価実験}
本章では,提案ツールの有効性を検証するために行った実験について述べる.

\section{仮説}
本実験では以下の仮説を検討する.

\begin{description}
    \item[仮説1] 提案ツールを使用することで,ドキュメントとソースコードの乖離を抑制できる
\end{description}

\section{実験計画}
本実験では,実験用に用意したプロジェクトに実験協力者が途中から参加し,ドキュメントとソースコードの追加・変更を行ってもらう.
このとき,実験協力者は前半5個,後半5個の合計10個のタスクをこなし,前半で提案ツールを使用する郡と後半で提案ツールを使用する郡に分けた.
また,実験後にはアンケートに回答してもらう.
ドキュメントとソースコードが乖離,アンケート結果,コミット履歴から提案ツールの有効性の検証を行う.

\subsection{実験協力者}
本実験での実験協力者は,静岡大学総合科学技術研究科情報学専攻の学生2名で,両名ともPythonとGitを利用することができる.
評価実験は,1名は2020年12月26日と12月27日に実験を行い,もう1名は1月6日と1月7日に実施した.
本実験では,前半に提案ツールを使用し,後半に提案ツールをしない1名と,前半に提案ツールを使用せず,後半に提案ツールを使用する1名に振り分けた.

\subsection{実験で使用したプロジェクト}
本実験で使用したプロジェクトを付録Aに示す.
実験では,前半5個,後半5個の合計10個のタスクを用意しており,各タスクではソースコードとドキュメントの追加や修正を行う必要がある.
また,前後半で難易度に大きな差がでないように調整した.

\subsection{実験の流れ}
一人目の実験協力者は下記の流れで実験を行った.
\begin{enumerate}
    \item 本実験の内容の説明
    \item 前半タスク (提案ツールを使用)
    \item 後半タスク (提案ツールを使用しない)
    \item 事後アンケート
\end{enumerate}

また,二人目の実験協力者は下記の流れで実験を行った.
\begin{enumerate}
    \item 本実験の内容の説明
    \item 前半タスク (提案ツールを使用しない)
    \item 後半タスク (提案ツールを使用)
    \item 事後アンケート
\end{enumerate}

実験協力者は,各タスクを終えるたびに変更履歴をコミットし,リモートリポジトリであるGitHubにプッシュする.
提案ツールを使用している場合,ドキュメントとソースコードで乖離を検知するとSlackから通知を受け取ることができる.
また,実験では実際のソフトウェア開発現場で途中から参加した開発者であることを想定し,
実験協力者がタスクに取り組むにあたって技術的にわからないことがあったときは,いつでも質問してよいこととした.