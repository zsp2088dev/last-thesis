\chapter{結論}
本研究では,ドキュメントとソースコードの乖離の可能性のある箇所の特定および通知を行うことのできるツールの開発を行った.
ドキュメントとソースコードが乖離してしまう原因は,ビジネス要求を素早く実現することが求めらるソフトウェア開発現場において,
ソースコードの追加・修正に追われてしまい,ドキュメントの追加・修正を忘れてしまうことにあると考えた.
そこで,ドキュメントとソフトウェアの乖離の可能性のある箇所を特定および通知を行い,
ドキュメントの追加・修正を開発者に促すことでドキュメントとソースコードの乖離を抑制できると考えた.
また,開発スピードを落とさないためにも,ライブラリやフレームワーク,エディタに依存せずに,既存のソフトウェア開発プロジェクトに無理なく組み込むことができる必要があると考えた.

今回提案したツールの効果を検証するために,Gitのコミット履歴を用いたシミュレーションおよび実験協力者に提案ツールを使用してもらい実験を行った.
シミュレーションの結果から〜〜.

実験では,前半5個,後半5個の合計10個のタスクを2名の実験協力者にこなしてもらい,前後半のいずれかで提案ツールが動作するように設定した.
実験の結果,前半に提案ツールを使用した実験協力者は,前半のタスクを2つこなした後に,提案ツールから乖離検知の通知を受け取った.
この後,実験協力者がドキュメントの修正を行ったため,前半タスク終了時にはドキュメントとソースコードの乖離が抑制された.
また,提案ツールを使用しない後半タスク終了時には,前半タスクで乖離したことを踏まえて作業を行ったため,ドキュメントとソースコードが乖離することはなかった.
後半に提案ツールを使用した実験協力者は,前半タスク終了時までにドキュメントを追加・修正することがなかった.
しかし,後半タスク開始時に乖離検知の通知を受け取った後に,ドキュメントの修正を行い,後半タスク終了時にはドキュメントとソースコードが乖離することはなかった.
実験協力者2名の実験終了後にドキュメントとソースコードが乖離していなかったことから,提案ツールはドキュメントとソースコードの乖離を抑制するのに一定の効果があることが判明した.

今後の課題として,様々なプログラミング言語の対応や,汎用性の向上,より正確な乖離リスクの特定を行う必要がある.