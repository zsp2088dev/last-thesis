\begin{thebibliography}{9}
    \bibitem{seigousei} 赤石裕里花,坂井麻里恵,奥野拓,伊藤恵:整合性維持に着目したソースコードとドキュメントの一元管理環境の提案,日本ソフトウェア科学会第30回大会,2013
    \bibitem{taiouduke} 後藤英斗,大久保弘崇,粕谷英人,山本晋一郎:文脈に基づいたソースプログラムとドキュメント間の識別子対応付け手法,商法処理学会研究報告,2005
    \bibitem{framework} 小池晃弘:ソースコードを記述しないアプリケーションフレームワークの提案,情報処理学会第76回全国大会,2014
    \bibitem{webcomponents} 海老澤雄太,丸山一貴,寺田実:Web Components開発におけるドキュメント同時生成手法の提案,第57回プログラミング・シンポジウム,2016
    \bibitem{test} 丹野治門,張暁晶:ソフトウェア設計ドキュメントを利用したテスト実行スクリプト生成技術の提案と評価,情報処理学会研究報告,2015
    \bibitem{slack} Slack, https://slack.com/intl/ja-jp/, (最終閲覧日:2021年1月18日)
    \bibitem{actions} GitHub Actions, https://github.co.jp/features/actions, (最終閲覧日:2021年1月18日)
    \bibitem{javadoc} Javadoc, https://docs.oracle.com/javase/jp/8/docs/technotes/tools/windows/javadoc.html, (最終閲覧日:2021年1月18日)
    \bibitem{spring} Spring Boot, https://spring.io/projects/spring-boot, (最終閲覧日:2021年1月18日)
    \bibitem{swagger} Swagger, https://swagger.io/, (最終閲覧日:2021年1月18日)
    \bibitem{redoc} ReDoc, https://github.com/Redocly/redoc, (最終閲覧日:2021年1月18日)
    \bibitem{fastapi} FastAPI, https://fastapi.tiangolo.com/, (最終閲覧日:2021年1月18日)
    \bibitem{gcp} Google Cloud Platform, https://console.cloud.google.com/, (最終閲覧日:2021年1月18日)
    \bibitem{heroku} Heroku, https://jp.heroku.com/, (最終閲覧日:2021年1月18日)
    \bibitem{matplotlib} Matplotlib, https://matplotlib.org/, (最終閲覧日:2021年1月18日)
    \bibitem{ngrok} ngrok, https://ngrok.com/, (最終閲覧日:2021年1月18日)
    \bibitem{pipenv} Pipenv, https://github.com/pypa/pipenv, (最終閲覧日:2021年1月18日)
    \bibitem{pillow} Pillow, https://github.com/python-pillow/Pillow, (最終閲覧日:2021年1月18日)
\end{thebibliography}